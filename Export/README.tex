\documentclass[]{article}
\usepackage{lmodern}
\usepackage{amssymb,amsmath}
\usepackage{ifxetex,ifluatex}
\usepackage{fixltx2e} % provides \textsubscript
\ifnum 0\ifxetex 1\fi\ifluatex 1\fi=0 % if pdftex
  \usepackage[T1]{fontenc}
  \usepackage[utf8]{inputenc}
\else % if luatex or xelatex
  \ifxetex
    \usepackage{mathspec}
  \else
    \usepackage{fontspec}
  \fi
  \defaultfontfeatures{Ligatures=TeX,Scale=MatchLowercase}
\fi
% use upquote if available, for straight quotes in verbatim environments
\IfFileExists{upquote.sty}{\usepackage{upquote}}{}
% use microtype if available
\IfFileExists{microtype.sty}{%
\usepackage{microtype}
\UseMicrotypeSet[protrusion]{basicmath} % disable protrusion for tt fonts
}{}
\usepackage[margin=1in]{geometry}
\usepackage{hyperref}
\hypersetup{unicode=true,
            pdftitle={Trapped Charge Data Analysis: Task View},
            pdfauthor={Sebastian Kreutzer},
            pdfborder={0 0 0},
            breaklinks=true}
\urlstyle{same}  % don't use monospace font for urls
\usepackage{longtable,booktabs}
\usepackage{graphicx,grffile}
\makeatletter
\def\maxwidth{\ifdim\Gin@nat@width>\linewidth\linewidth\else\Gin@nat@width\fi}
\def\maxheight{\ifdim\Gin@nat@height>\textheight\textheight\else\Gin@nat@height\fi}
\makeatother
% Scale images if necessary, so that they will not overflow the page
% margins by default, and it is still possible to overwrite the defaults
% using explicit options in \includegraphics[width, height, ...]{}
\setkeys{Gin}{width=\maxwidth,height=\maxheight,keepaspectratio}
\IfFileExists{parskip.sty}{%
\usepackage{parskip}
}{% else
\setlength{\parindent}{0pt}
\setlength{\parskip}{6pt plus 2pt minus 1pt}
}
\setlength{\emergencystretch}{3em}  % prevent overfull lines
\providecommand{\tightlist}{%
  \setlength{\itemsep}{0pt}\setlength{\parskip}{0pt}}
\setcounter{secnumdepth}{0}
% Redefines (sub)paragraphs to behave more like sections
\ifx\paragraph\undefined\else
\let\oldparagraph\paragraph
\renewcommand{\paragraph}[1]{\oldparagraph{#1}\mbox{}}
\fi
\ifx\subparagraph\undefined\else
\let\oldsubparagraph\subparagraph
\renewcommand{\subparagraph}[1]{\oldsubparagraph{#1}\mbox{}}
\fi

%%% Use protect on footnotes to avoid problems with footnotes in titles
\let\rmarkdownfootnote\footnote%
\def\footnote{\protect\rmarkdownfootnote}

%%% Change title format to be more compact
\usepackage{titling}

% Create subtitle command for use in maketitle
\providecommand{\subtitle}[1]{
  \posttitle{
    \begin{center}\large#1\end{center}
    }
}

\setlength{\droptitle}{-2em}

  \title{Trapped Charge Data Analysis: Task View}
    \pretitle{\vspace{\droptitle}\centering\huge}
  \posttitle{\par}
    \author{Sebastian Kreutzer}
    \preauthor{\centering\large\emph}
  \postauthor{\par}
      \predate{\centering\large\emph}
  \postdate{\par}
    \date{(last update: 2019-06-10)}

\usepackage{amsmath}
\usepackage{amssymb}

\begin{document}
\maketitle

\hypertarget{about}{%
\section{About}\label{about}}

In analogy of the \href{https://cran.rstudio.com/web/views/}{CRAN task view lists},
this list provides an overview of freely available tools for trapped charge (dating) data analysis (e.g., luminescence and ESR data). Tasks, software tools and data repositories are ordered alphabetically. URLs are automatically tested every time this list is updated, information from R packages are extracted and updated automatically from \href{https://cloud.r-project.org}{CRAN} (version, description).

Not listed are:

\begin{itemize}
\tightlist
\item
  Single scripts (e.g., functions or XLS-sheets usually without a dedicated name)
\item
  Software not accessible free of charge through the internet
\end{itemize}

If \textbf{your software is missing} or you did spot a mistake, please let me know via \url{https://github.com/RLumSK/luminescence-tv/issues}.

\hypertarget{mission-tasks}{%
\section{Mission tasks}\label{mission-tasks}}

Total number of listed tools: 35

\hypertarget{age-calculation}{%
\subsubsection{Age calculation}\label{age-calculation}}

\begin{itemize}
\tightlist
\item
  AGE {[}2009-11-14{]}~~~\includegraphics{https://img.shields.io/badge/-WIN-lightgrey.svg}~
  Program for the calculation of luminescence ages estimates
  \includegraphics{https://img.shields.io/badge/-URL-lightgrey.svg}~\url{http://ancienttl.org/ATL_27-2_2009/ATL_27-2_Grun_supplement.zip}
  \emph{\url{http://ancienttl.org/ATL_27-2_2009/ATL_27-2_Grun_p45-46.pdf}}
\item
  DRAC {[}1.2{]}~~~~\includegraphics{https://img.shields.io/badge/-WIN-lightgrey.svg}~\includegraphics{https://img.shields.io/badge/-MAC-lightgrey.svg}~\includegraphics{https://img.shields.io/badge/-LIN-lightgrey.svg}~
  Online dose rate and luminescence age calculator
  \includegraphics{https://img.shields.io/badge/-URL-lightgrey.svg}~\url{http://www.aber.ac.uk/en/dges/research/quaternary/luminescence-research-laboratory/dose-rate-calculator/}
  \emph{Durcan, J.A., King, G.E., Duller, G.A.T., 2015. DRAC: Dose Rate and Age Calculator for trapped charge dating. Quaternary Geochronology 28, 54--61. doi: \url{https://doi.org/10.1016/j.quageo.2015.03.012}}
\item
  DRc ~~\includegraphics{https://img.shields.io/badge/-WIN-lightgrey.svg}~
  Dose Rate Calculator for luminescence and ESR Dating (Java application)
  \includegraphics{https://img.shields.io/badge/-URL-red.svg}~\url{http://www.ims.demokritos.gr/download/}
  \emph{Tsakalos, E., Christodoulakis, J., Charalambous, L., 2015. The Dose Rate Calculator (DRc) for Luminescence and ESR Dating-a Java Application for Dose Rate and Age Determination. Archaeometry 58, 347--352. doi: \url{https://doi.org/10.1111/arcm.12162}}
\end{itemize}

\hypertarget{chronological-modelling}{%
\subsubsection{Chronological modelling}\label{chronological-modelling}}

\begin{itemize}
\tightlist
\item
  ArchaeoChron {[}0.1{]}~~~~\includegraphics{https://img.shields.io/badge/-WIN-lightgrey.svg}~\includegraphics{https://img.shields.io/badge/-MAC-lightgrey.svg}~\includegraphics{https://img.shields.io/badge/-LIN-lightgrey.svg}~
  Provides a list of functions for the Bayesian modeling of archaeological chronologies. The Bayesian models are implemented in `JAGS' (`JAGS' stands for Just Another Gibbs Sampler. It is a program for the analysis of Bayesian hierarchical models using Markov Chain Monte Carlo (MCMC) simulation. See \url{http://mcmc-jags.sourceforge.net/} and ``JAGS Version 4.3.0 user manual'', Martin Plummer (2017) \url{https://sourceforge.net/projects/mcmc-jags/files/Manuals/}.). The inputs are measurements with their associated standard deviations and the study period. The output is the MCMC sample of the posterior distribution of the event date with or without radiocarbon calibration.
  \includegraphics{https://img.shields.io/badge/-URL-lightgrey.svg}~\url{https://CRAN.R-project.org/package=ArchaeoChron}
\item
  ArchaeoPhases {[}1.4{]}~~~~\includegraphics{https://img.shields.io/badge/-WIN-lightgrey.svg}~\includegraphics{https://img.shields.io/badge/-MAC-lightgrey.svg}~\includegraphics{https://img.shields.io/badge/-LIN-lightgrey.svg}~
  Provides a list of functions for the statistical analysis of archaeological dates and groups of dates. It is based on the post-processing of the Markov Chains whose stationary distribution is the posterior distribution of a series of dates. Such output can be simulated by different applications as for instance `ChronoModel' (see \url{http://www.chronomodel.fr}), `Oxcal' (see \url{https://c14.arch.ox.ac.uk/oxcal.html}) or `BCal' (see \url{http://bcal.shef.ac.uk/}). The only requirement is to have a csv file containing a sample from the posterior distribution.
  \includegraphics{https://img.shields.io/badge/-URL-lightgrey.svg}~\url{https://CRAN.R-project.org/package=ArchaeoPhases}
  \emph{Philippe, A., Vibet, M.-A., 2018. Analysis of Archaeological Phases using the CRAN Package ArchaeoPhases. Journal of Statistical Software 1--26. doi: \url{https://doi.org/10.18637/jss.v000.i00}}
\item
  BayLum {[}0.1.3{]}~~~~\includegraphics{https://img.shields.io/badge/-WIN-lightgrey.svg}~\includegraphics{https://img.shields.io/badge/-MAC-lightgrey.svg}~\includegraphics{https://img.shields.io/badge/-LIN-lightgrey.svg}~
  Bayesian analysis of luminescence data and C-14 age estimates. Bayesian models are based on the following publications: Combes, B. \& Philippe, A. (2017) \url{doi:10.1016/j.quageo.2017.02.003} and Combes et al (2015) \url{doi:10.1016/j.quageo.2015.04.001}. This includes, amongst others, data import, export, application of age models and palaeodose model.
  \includegraphics{https://img.shields.io/badge/-URL-lightgrey.svg}~\url{https://CRAN.R-project.org/package=BayLum}
  \emph{Philippe, A., Guérin, G., Kreutzer, S., 2019. BayLum - An R package for Bayesian analysis of OSL ages: An introduction. Quaternary Geochronology 49, 16--24. doi: \url{https://doi.org/10.1016/j.quageo.2018.05.009}}
\item
  ChronoModel {[}2.0.18{]}~{[}2019-02-01{]}~~~~\includegraphics{https://img.shields.io/badge/-WIN-lightgrey.svg}~\includegraphics{https://img.shields.io/badge/-MAC-lightgrey.svg}~\includegraphics{https://img.shields.io/badge/-LIN-lightgrey.svg}~
  Chronological Modelling of Archaeological Data using Bayesian Statistics with an advanced graphical user interface
  \includegraphics{https://img.shields.io/badge/-URL-lightgrey.svg}~\url{https://chronomodel.com}
  \textbar{} Source code: \url{https://github.com/Chronomodel/chronomodel}
\item
  mcmcSAM {[}Mon, 07 Jan 2019 14:17:22 GMT{]}~~~~\includegraphics{https://img.shields.io/badge/-WIN-lightgrey.svg}~\includegraphics{https://img.shields.io/badge/-MAC-lightgrey.svg}~\includegraphics{https://img.shields.io/badge/-LIN-lightgrey.svg}~
  Analyzing statistical age models for equivalent dose and burial age determination using a Markov Chain Monte Carlo method
  \includegraphics{https://img.shields.io/badge/-URL-lightgrey.svg}~\url{https://github.com/pengjunUCAS/mcmcSAM}
\item
  RChronoModel {[}0.4{]}~~~~\includegraphics{https://img.shields.io/badge/-WIN-lightgrey.svg}~\includegraphics{https://img.shields.io/badge/-MAC-lightgrey.svg}~
  Provides a list of functions for the statistical analysis and the post-processing of the Markov Chains simulated by ChronoModel (see \url{http://www.chronomodel.fr} for more information). ChronoModel is a friendly software to construct a chronological model in a Bayesian framework. Its output is a sampled Markov chain from the posterior distribution of dates component the chronology. The functions can also be applied to the analyse of mcmc output generated by Oxcal software.
  \includegraphics{https://img.shields.io/badge/-URL-lightgrey.svg}~\url{https://CRAN.R-project.org/package=RChronoModel}
  \emph{Philippe, A., Vibet, M.-A., 2017. Analysis of Archaeological Phases using the CRAN Package RChronoModel. doi: \url{https://doi.org/10.13140/RG.2.2.19659.59688}}
\end{itemize}

\hypertarget{data-mining}{%
\subsubsection{Data mining}\label{data-mining}}

\begin{itemize}
\tightlist
\item
  INQUA Dunes Atlas ~~\includegraphics{https://img.shields.io/badge/-WIN-lightgrey.svg}~\includegraphics{https://img.shields.io/badge/-MAC-lightgrey.svg}~\includegraphics{https://img.shields.io/badge/-LIN-lightgrey.svg}~
  Collection of luminescence ages from sand dunes world wide
  \includegraphics{https://img.shields.io/badge/-URL-lightgrey.svg}~\url{http://www.dri.edu/inquadunesatlas}
  \emph{Lancaster, N., Wolfe, S., Thomas, D., Bristow, C., Bubenzer, O., Burrough, S., Duller, G., Halfen, A., Hesse, P., Roskin, J., Singhvi, A., Tsoar, H., Tripaldi, A., Yang, X., Zárate, M., 2015. The INQUA Dunes Atlas chronologic database. Quaternary International 410, 3--10. doi: \url{https://doi.org/10.1016/j.quaint.2015.10.044}}
\item
  OCTOPUS ~~\includegraphics{https://img.shields.io/badge/-WIN-lightgrey.svg}~\includegraphics{https://img.shields.io/badge/-MAC-lightgrey.svg}~\includegraphics{https://img.shields.io/badge/-LIN-lightgrey.svg}~
  Open cosmogenic nuclide and luminescence data database
  \includegraphics{https://img.shields.io/badge/-URL-lightgrey.svg}~\url{https://earth.uow.edu.au}
  \emph{Codilean, A.T., Munack, H., Cohen, T.J., Saktura, W.M., Gray, A., Mudd, S.M., 2018. OCTOPUS: an open cosmogenic isotope and luminescence database. Earth Syst. Sci. Data 10, 2123--2139. doi: \url{https://doi.org/10.5194/essd-10-2123-2018}}
\end{itemize}

\hypertarget{dose-rate-modelling}{%
\subsubsection{Dose rate modelling}\label{dose-rate-modelling}}

\begin{itemize}
\tightlist
\item
  DosiVox {[}2018-12-03{]}~~~~\includegraphics{https://img.shields.io/badge/-WIN-lightgrey.svg}~\includegraphics{https://img.shields.io/badge/-MAC-lightgrey.svg}~\includegraphics{https://img.shields.io/badge/-LIN-lightgrey.svg}~
  A Geant 4-based software for dosimetry simulations relevant to luminescence and ESR dating techniques
  \includegraphics{https://img.shields.io/badge/-URL-lightgrey.svg}~\url{http://www.iramat-crp2a.cnrs.fr/spip/spip.php?article144\&lang=fr}
  \emph{Martin, L., Incerti, S., Mercier, N., 2015. DosiVox: Implementing Geant 4-based software for dosimetry simulations relevant to luminescence and ESR dating techniques. Ancient TL 33, 1--10. \url{http://ancienttl.org/ATL_33-1_2015/ATL_33-1_Martin_p1-10.pdf}}
\item
  DosiVox2D {[}2018-12-03{]}~~~~\includegraphics{https://img.shields.io/badge/-WIN-lightgrey.svg}~\includegraphics{https://img.shields.io/badge/-MAC-lightgrey.svg}~\includegraphics{https://img.shields.io/badge/-LIN-lightgrey.svg}~
  A Geant 4-based software for dosimetry simulations relevant to luminescence and ESR dating techniques; simplified version in comparison to DosiVox
  \includegraphics{https://img.shields.io/badge/-URL-lightgrey.svg}~\url{http://www.iramat-crp2a.cnrs.fr/spip/spip.php?article144\&lang=fr}
\item
  RCarb {[}0.1.3{]}~~~~\includegraphics{https://img.shields.io/badge/-WIN-lightgrey.svg}~\includegraphics{https://img.shields.io/badge/-MAC-lightgrey.svg}~\includegraphics{https://img.shields.io/badge/-LIN-lightgrey.svg}~
  Translation of the `MATLAB' program `Carb' (Nathan and Mauz 2008 \url{doi:10.1016/j.radmeas.2007.12.012}; Mauz and Hoffmann 2014) for dose rate modelling for carbonate-rich samples in the context of trapped charged dating (e.g., luminescence dating) applications.
  \includegraphics{https://img.shields.io/badge/-URL-lightgrey.svg}~\url{https://CRAN.R-project.org/package=RCarb}
\end{itemize}

\hypertarget{dosimetry}{%
\subsubsection{Dosimetry}\label{dosimetry}}

\begin{itemize}
\tightlist
\item
  gamma {[}0.1{]}~{[}2019-04-26{]}~~~~\includegraphics{https://img.shields.io/badge/-WIN-lightgrey.svg}~\includegraphics{https://img.shields.io/badge/-MAC-lightgrey.svg}~\includegraphics{https://img.shields.io/badge/-LIN-lightgrey.svg}~
  gamma is intended to process in-situ gamma-ray spectrometry measurements for luminescence dating. This package allows to import, inspect and (automatically) correct the energy scale of the spectrum. It provides methods for estimating the gamma dose rate by the use of a calibration curve. This package only supports Canberra CNF and TKA files.
  \includegraphics{https://img.shields.io/badge/-URL-lightgrey.svg}~\url{http://gamma.archaeo.science}
  \textbar{} Source code: \url{https://github.com/crp2a/gamma}
  \href{https://doi.org/10.5281/zenodo.2652393}{\includegraphics{https://img.shields.io/static/v1.svg?style=flat\&label=DOI\&message=10.5281/zenodo.2652393\&color=blue}}
  \emph{Frerebeau, N, Lebrun, B., Guérin, G., Lahaye, C., 2019. gamma: Dose Rate Estimation from In-Situ Gamma-Ray Spectrometry Measurements for Luminescence Dating. \url{http://gamma.archaeo.science}, \url{http://github.com/crp2a/gamma}.}
\end{itemize}

\hypertarget{esr-data-analysis}{%
\subsubsection{ESR data analysis}\label{esr-data-analysis}}

\begin{itemize}
\tightlist
\item
  ESR {[}Fri, 18 Jan 2019 12:31:54 GMT{]}~~~~\includegraphics{https://img.shields.io/badge/-WIN-lightgrey.svg}~\includegraphics{https://img.shields.io/badge/-MAC-lightgrey.svg}~\includegraphics{https://img.shields.io/badge/-LIN-lightgrey.svg}~
  R package ESR for plotting and analysing ESR spectra in dating applications
  \includegraphics{https://img.shields.io/badge/-URL-lightgrey.svg}~\url{https://github.com/tzerk/ESR}
\item
  MCDoseE ~~\includegraphics{https://img.shields.io/badge/-WIN-lightgrey.svg}~\includegraphics{https://img.shields.io/badge/-MAC-lightgrey.svg}~
  Dose response curve fitting, dose evaluation for ESR dating
  \includegraphics{https://img.shields.io/badge/-URL-lightgrey.svg}~\url{https://www.sciencedirect.com/science/article/pii/S1871101417300626?via\%3Dihub}
  \emph{Joannes-Boyau, R., Duval, M., Bodin, T., 2017. MCDoseE 2.0 A new Markov Chain Monte Carlo program for ESR dose response curve fitting and dose evaluation. Quaternary Geochronology 44, 1--25. doi: \url{https://doi.org/10.1016/j.quageo.2017.11.003}}
\end{itemize}

\hypertarget{gamma-ray-spectrometry}{%
\subsubsection{Gamma-ray spectrometry}\label{gamma-ray-spectrometry}}

\begin{itemize}
\tightlist
\item
  gammaSpec {[}Fri, 18 Jan 2019 15:13:50 GMT{]}~~~~\includegraphics{https://img.shields.io/badge/-WIN-lightgrey.svg}~\includegraphics{https://img.shields.io/badge/-MAC-lightgrey.svg}~\includegraphics{https://img.shields.io/badge/-LIN-lightgrey.svg}~
  A collection of functions to analyse gamma-ray spectra
  \includegraphics{https://img.shields.io/badge/-URL-lightgrey.svg}~\url{https://github.com/tzerk/gammaSpec}
\item
  rxylib {[}0.2.4{]}~~~~\includegraphics{https://img.shields.io/badge/-WIN-lightgrey.svg}~\includegraphics{https://img.shields.io/badge/-MAC-lightgrey.svg}~\includegraphics{https://img.shields.io/badge/-LIN-lightgrey.svg}~
  Provides access to the `xylib' C library for to import xy
  data from powder diffraction, spectroscopy and other experimental methods.
  \includegraphics{https://img.shields.io/badge/-URL-red.svg}~\url{https://CRAN.R-project.org/package=rxylib}
\end{itemize}

\hypertarget{luminescence-data-analysis}{%
\subsubsection{Luminescence data analysis}\label{luminescence-data-analysis}}

\begin{itemize}
\tightlist
\item
  Analyst {[}4.57{]}~~~\includegraphics{https://img.shields.io/badge/-WIN-lightgrey.svg}~
  The standard programme to analyse luminescence data
  \includegraphics{https://img.shields.io/badge/-URL-lightgrey.svg}~\url{http://users.aber.ac.uk/ggd/}
  \emph{Duller, G.A.T., 2015. The Analyst software package for luminescence data: overview and recent improvements. Ancient TL 33, 35--42. \url{http://ancienttl.org/ATL_33-1_2015/ATL_33-1_Duller_p35-42.pdf}}
\item
  Luminescence {[}0.9.0.110{]}~~~~\includegraphics{https://img.shields.io/badge/-WIN-lightgrey.svg}~\includegraphics{https://img.shields.io/badge/-MAC-lightgrey.svg}~\includegraphics{https://img.shields.io/badge/-LIN-lightgrey.svg}~
  A collection of various R functions for the purpose of Luminescence
  dating data analysis. This includes, amongst others, data import, export,
  application of age models, curve deconvolution, sequence analysis and
  plotting of equivalent dose distributions.
  \includegraphics{https://img.shields.io/badge/-URL-lightgrey.svg}~\url{https://CRAN.R-project.org/package=Luminescence}
  \emph{Kreutzer, S., Schmidt, C., Fuchs, M.C., Dietze, M., Fischer, M., Fuchs, M., 2012. Introducing an R package for luminescence dating analysis. Ancient TL 30, 1--8. \url{http://ancienttl.org/ATL_30-1_2012/ATL_30-1_Kreutzer_p1-8.pdf}}
\item
  numOSL {[}2.6{]}~~~~\includegraphics{https://img.shields.io/badge/-WIN-lightgrey.svg}~\includegraphics{https://img.shields.io/badge/-MAC-lightgrey.svg}~\includegraphics{https://img.shields.io/badge/-LIN-lightgrey.svg}~
  Package for optimizing regular numeric problems in optically stimulated luminescence
  dating, such as: equivalent dose calculation, dose rate determination, growth curve fitting,
  decay curve decomposition, statistical age model optimization, and statistical plot visualization.
  \includegraphics{https://img.shields.io/badge/-URL-lightgrey.svg}~\url{https://CRAN.R-project.org/package=numOSL}
  \emph{Peng, J., Dong, Z., Han, F., Long, H., Liu, X., 2013. R package numOSL: numeric routines for optically stimulated luminescence dating. Ancient TL 31, 41--48. \url{http://ancienttl.org/ATL_31-2_2013/ATL_31-2_Peng_p41-48.pdf}}
\item
  PTanalyse {[}1.51{]}~~~\includegraphics{https://img.shields.io/badge/-WIN-lightgrey.svg}~
  Proprietary software to analyse TR-OSL data
  \includegraphics{https://img.shields.io/badge/-URL-lightgrey.svg}~\url{https://www.nutech.dtu.dk/english/products-and-services/radiation-instruments/tl_osl_reader/software}
  \emph{Lapp, T., Jain, M., Ankjærgaard, C., Pirtzel, L., 2009. Development of pulsed stimulation and Photon Timer attachments to the Risø TL/OSL reader. Radiation Measurements 44, 571--575. doi: \url{https://doi.org/10.1016/j.radmeas.2009.01.012}}
\item
  RLanalyse {[}1.3{]}~~~\includegraphics{https://img.shields.io/badge/-WIN-lightgrey.svg}~
  Proprietary software to analyse radiofluorescence data (BIN/BINX-file input required)
  \includegraphics{https://img.shields.io/badge/-URL-lightgrey.svg}~\url{https://www.nutech.dtu.dk/english/products-and-services/radiation-instruments/tl_osl_reader/software}
  \emph{Lapp, T., Jain, M., Thomsen, K.J., Murray, A.S., Buylaert, J.P., 2012. New luminescence measurement facilities in retrospective dosimetry. Radiation Measurements 47, 803--808. doi: \url{https://doi.org/10.1016/j.radmeas.2012.02.006}}
\item
  tgcd {[}2.1{]}~~~~\includegraphics{https://img.shields.io/badge/-WIN-lightgrey.svg}~\includegraphics{https://img.shields.io/badge/-MAC-lightgrey.svg}~\includegraphics{https://img.shields.io/badge/-LIN-lightgrey.svg}~
  Deconvolving thermoluminescence glow curves according to various
  kinetic models (first-order, second-order, general-order, and mixed-order) using
  a modified Levenberg-Marquardt algorithm. It provides the possibility of setting
  constraints or fixing any of parameters. It offers an interactive way to initialize
  parameters by clicking with a mouse on a plot at positions where peak maxima should
  be located. The optimal estimate is obtained by ``trial-and-error''. It also provides
  routines for simulating first-order, second-order, and general-order glow peaks.
  \includegraphics{https://img.shields.io/badge/-URL-lightgrey.svg}~\url{https://CRAN.R-project.org/package=tgcd}
  \emph{Peng, J., Dong, Z., Han, F., 2016. tgcd: An R package for analyzing thermoluminescence glow curves. SoftwareX 1--9. doi: \url{https://doi.org/10.1016/j.softx.2016.06.001}}
\item
  TLdating {[}0.1.3{]}~~~~\includegraphics{https://img.shields.io/badge/-WIN-lightgrey.svg}~\includegraphics{https://img.shields.io/badge/-MAC-lightgrey.svg}~\includegraphics{https://img.shields.io/badge/-LIN-lightgrey.svg}~
  A series of function to make thermoluminescence dating using the MAAD or the SAR protocol.
  This package completes the R package ``Luminescence.''
  \includegraphics{https://img.shields.io/badge/-URL-lightgrey.svg}~\url{https://CRAN.R-project.org/package=TLdating}
  \emph{Strebler, D., Burow, C., Brill, D., Brückner, H., 2017. Using R for TL dating. Quaternary Geochronology 37, 97--107. doi: \url{https://doi.org/10.1016/j.quageo.2016.09.001}}
\end{itemize}

\hypertarget{luminescence-data-visualisation}{%
\subsubsection{Luminescence data visualisation}\label{luminescence-data-visualisation}}

\begin{itemize}
\tightlist
\item
  Viewer {[}4.4{]}~~~\includegraphics{https://img.shields.io/badge/-WIN-lightgrey.svg}~
  Proprietary software to visualise luminescence data recorded in BIN/BINX-files
  \includegraphics{https://img.shields.io/badge/-URL-lightgrey.svg}~\url{https://www.nutech.dtu.dk/english/products-and-services/radiation-instruments/tl_osl_reader/software}
  \emph{ }
\item
  Viewer+ {[}1.43{]}~~~\includegraphics{https://img.shields.io/badge/-WIN-lightgrey.svg}~
  Proprietary software to visualise luminescence data recorded in BIN/BINX-files; sucessor of Viewer
  \includegraphics{https://img.shields.io/badge/-URL-lightgrey.svg}~\url{https://www.nutech.dtu.dk/english/products-and-services/radiation-instruments/tl_osl_reader/software}
  \emph{ }
\end{itemize}

\hypertarget{miscellaneous}{%
\subsubsection{Miscellaneous}\label{miscellaneous}}

\begin{itemize}
\tightlist
\item
  XRFanalyse {[}1.12{]}~~~\includegraphics{https://img.shields.io/badge/-WIN-lightgrey.svg}~
  Proprietary software to analyse XRF data recorded with a Risø OSL/TL attachement
  \includegraphics{https://img.shields.io/badge/-URL-lightgrey.svg}~\url{https://www.nutech.dtu.dk/english/products-and-services/radiation-instruments/tl_osl_reader/software}
  \emph{ }
\end{itemize}

\hypertarget{modelling}{%
\subsubsection{Modelling}\label{modelling}}

\begin{itemize}
\tightlist
\item
  KMS {[}Wed, 11 Jul 2018 01:41:12 GMT{]}~~~~\includegraphics{https://img.shields.io/badge/-WIN-lightgrey.svg}~\includegraphics{https://img.shields.io/badge/-MAC-lightgrey.svg}~\includegraphics{https://img.shields.io/badge/-LIN-lightgrey.svg}~
  Collection of functions to simulate kinetic models for quartz luminescence production
  \includegraphics{https://img.shields.io/badge/-URL-lightgrey.svg}~\url{https://github.com/pengjunUCAS/KMS}
  \emph{Peng, J., Pagonis, V., 2016. Simulating comprehensive kinetic models for quartz luminescence using the R program KMS. Radiation Measurements 86, 63--70. doi: \url{https://doi.org/10.1016/j.radmeas.2016.01.022}}
\item
  RLumModel {[}0.2.3{]}~~~~\includegraphics{https://img.shields.io/badge/-WIN-lightgrey.svg}~\includegraphics{https://img.shields.io/badge/-MAC-lightgrey.svg}~\includegraphics{https://img.shields.io/badge/-LIN-lightgrey.svg}~
  A collection of functions to simulate luminescence signals in quartz and Al2O3 based on published models.
  \includegraphics{https://img.shields.io/badge/-URL-lightgrey.svg}~\url{https://CRAN.R-project.org/package=RLumModel}
  \emph{Friedrich, J., Kreutzer, S., Schmidt, C., 2016. Solving ordinary differential equations to understand luminescence: ``RLumModel'' an advanced research tool for simulating luminescence in quartz using R. Quaternary Geochronology 35, 88--100. doi: \url{https://doi.org/10.1016/j.quageo.2016.05.004}}
\end{itemize}

\hypertarget{plotting}{%
\subsubsection{Plotting}\label{plotting}}

\begin{itemize}
\tightlist
\item
  RLumShiny {[}0.2.2{]}~~~~\includegraphics{https://img.shields.io/badge/-WIN-lightgrey.svg}~\includegraphics{https://img.shields.io/badge/-MAC-lightgrey.svg}~\includegraphics{https://img.shields.io/badge/-LIN-lightgrey.svg}~
  A collection of `shiny' applications for the R package
  `Luminescence'. These mainly, but not exclusively, include applications for
  plotting chronometric data from e.g.~luminescence or radiocarbon dating. It
  further provides access to bootstraps tooltip and popover functionality and
  contains the `jscolor.js' library with a custom `shiny' output binding.
  \includegraphics{https://img.shields.io/badge/-URL-lightgrey.svg}~\url{https://CRAN.R-project.org/package=RLumShiny}
  \emph{Burow, C., Kreutzer, S., Dietze, M., Fuchs, M.C., Fischer, M., Schmidt, C., Brückner, H., 2016. RLumShiny - A graphical user interface for the R Package 'Luminescence'. Ancient TL 34, 22--32. \url{http://ancienttl.org/ATL_34-2_2016/ATL_34-2_Burow_p22-32.pdf}}
\end{itemize}

\hypertarget{teaching}{%
\subsubsection{Teaching}\label{teaching}}

\begin{itemize}
\tightlist
\item
  LumReader {[}0.1.0{]}~{[}2017-01-27{]}~~~~\includegraphics{https://img.shields.io/badge/-WIN-lightgrey.svg}~\includegraphics{https://img.shields.io/badge/-MAC-lightgrey.svg}~\includegraphics{https://img.shields.io/badge/-LIN-lightgrey.svg}~
  A series of functions to estimate the detection windows of a luminescence reader based on the filters and the photomultiplier (PMT) selected. These functions also allow to simulate a luminescence experiment based on the thermoluminesce (TL) or the optically stimulated luminescence (OSL) properties of a material
  \includegraphics{https://img.shields.io/badge/-URL-lightgrey.svg}~\url{https://CRAN.R-project.org/package=LumReader}
\end{itemize}

\hypertarget{visualisation}{%
\subsubsection{Visualisation}\label{visualisation}}

\begin{itemize}
\tightlist
\item
  DensityPlotter {[}8.4{]}~~~\includegraphics{https://img.shields.io/badge/-WIN-lightgrey.svg}~\includegraphics{https://img.shields.io/badge/-MAC-lightgrey.svg}~\includegraphics{https://img.shields.io/badge/-LIN-lightgrey.svg}~
  Java application for Kernel Density Estimation plots
  \includegraphics{https://img.shields.io/badge/-URL-lightgrey.svg}~\url{https://www.ucl.ac.uk/~ucfbpve/densityplotter/}
  \emph{Vermeesch, P., 2012. On the visualisation of detrital age distributions. Chemical Geology, 312-313, 190-194, doi: \url{https://doi.org/10.1016/j.chemgeo.2012.04.021}}
\item
  RadialPlotter {[}9.4{]}~~~\includegraphics{https://img.shields.io/badge/-WIN-lightgrey.svg}~\includegraphics{https://img.shields.io/badge/-MAC-lightgrey.svg}~\includegraphics{https://img.shields.io/badge/-LIN-lightgrey.svg}~
  Java software to create radial plots
  \includegraphics{https://img.shields.io/badge/-URL-lightgrey.svg}~\url{https://www.ucl.ac.uk/~ucfbpve/radialplotter/}
  \emph{Vermeesch, P., 2009, RadialPlotter: a Java application for fission track, luminescence and other radial plots, Radiation Measurements, 44 (4), 409-410. doi: \url{https://doi.org/10.1016/j.radmeas.2009.05.003}}
\end{itemize}

\hypertarget{legend}{%
\section*{Legend}\label{legend}}
\addcontentsline{toc}{section}{Legend}

\begin{longtable}[]{@{}ll@{}}
\toprule
Icon & Meaning\tabularnewline
\midrule
\endhead
{[}0.1{]} & Indicates the latest available version, here 0.1\tabularnewline
& This software is open source\tabularnewline
& Type of software (here: R package)\tabularnewline
\bottomrule
\end{longtable}

Some of the badges are created using \url{https://shields.io}.


\end{document}
